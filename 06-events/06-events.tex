% MIT License
%
% Copyright (c) 2023 Aliaksei Bialiauski
%
% Permission is hereby granted, free of charge, to any person obtaining a copy
% of this software and associated documentation files (the "Software"), to deal
% in the Software without restriction, including without limitation the rights
% to use, copy, modify, merge, publish, distribute, sublicense, and/or sell
% copies of the Software, and to permit persons to whom the Software is
% furnished to do so, subject to the following conditions:
%
% The above copyright notice and this permission notice shall be included in all
% copies or substantial portions of the Software.
%
% THE SOFTWARE IS PROVIDED "AS IS", WITHOUT WARRANTY OF ANY KIND, EXPRESS OR
% IMPLIED, INCLUDING BUT NOT LIMITED TO THE WARRANTIES OF MERCHANTABILITY,
% FITNESS FOR A PARTICULAR PURPOSE AND NONINFRINGEMENT. IN NO EVENT SHALL THE
% AUTHORS OR COPYRIGHT HOLDERS BE LIABLE FOR ANY CLAIM, DAMAGES OR OTHER
% LIABILITY, WHETHER IN AN ACTION OF CONTRACT, TORT OR OTHERWISE, ARISING FROM,
% OUT OF OR IN CONNECTION WITH THE SOFTWARE OR THE USE OR OTHER DEALINGS IN THE
% SOFTWARE.

\documentclass{article}
\usepackage{..//cover}
\usepackage{..//slides}
\usepackage{..//inno}
\usepackage[normalem]{ulem}
\newcommand*\thetitle{Event-Driven Architecture}
\newcommand*\thesubtitle{and CQRS}
\begin{document}

    \plush{\defaultInnoTitlePage \innoDisclaimer}

    \innoToc

    \plush{\innoChapter[Event]{Event sourcing}}

    \plush{%
        \innoPic{0.8}{eda}
    }

    \subcrumbection{Def}
    \plush{%
        \innoQuote{fowler}{
            The fundamental idea of Event Sourcing is that of ensuring every change
            to the state of an application is captured in an event object,
            and that these event objects are themselves stored in the sequence
            they were applied for the same lifetime as the application state itself
        }{Martin Fowler}
    }

    \plush[5]{%
        \innoPic{0.5}{eventsourcing}\par
        ``Event sourcing - is the process of persisting business entities as a sequence of events.''
    }

    \plush{%
        \innoPic{0.8}{triangle}
    }

    \plush{%
        \innoSection{Event sourcing Example}
        \innoPic{0.6}{example}
    }

    \subcrumbection{Aggregate}
    \plush[5]{%
        \innoBanner{Aggregate pattern}
        \innoPic{0.7}{aggregate1}\par
        ``In Event sourcing an application state represented by simple graph of objects(events) called Aggregate''
    }

    \plush[5]{%
        \innoPic{0.7}{aggregate2}
    }

    \subcrumbection{Projection}
    \plush[5]{%
        \innoBanner{Projection pattern}
        \innoPic{0.6}{projection}\par
        ``In Event Sourcing, Projections (also known as View Models or Query Models) provide a view of the underlying event-based data model.
        Often they represent the logic of translating the source write model into the read model.
        They are used in both read models and write models.''
    }

    \subcrumbection{Saga}
    \plush[5]{%
        \innoBanner{Saga}
        \innoPic{1}{saga}
    }

    \plush{\innoChapter[CQRS]{Command Query Responsibility Segregation (CQRS)}}

    \plush{%
        \innoPic{0.9}{cqrs1}
    }

    \plush{%
        \innoPic{0.8}{cqrs2}
    }

    \plush[5]{%
        \innoBanner{How to synchronize READ and WRITE model?}
        \begin{itemize}
            \item Database trigger
            \item Message queue
        \end{itemize}
    }

    \plush{\innoChapter[Properties]{Event-Driven Architecture properties}}

    \subcrumbection{Store}
    \plush[5]{%
        \innoBanner{Event Store}
        \innoPic{0.7}{store}\par
        ``Event Store contains all events. These events can be fetched to build Event Projection''
    }

    \subcrumbection{Bus}
    \plush[5]{%
        \innoBanner{Event Bus}
        \innoPic{0.6}{bus}\par
        ``An event bus is a mediator that transfers a message from a sender to a receiver.
        In this way, it provides a loosely coupled communication way between objects, services and applications.''
    }

    \plush[5]{%
        \url{https://kafka.apache.org}\par
        \url{https://aws.amazon.com/eventbridge}
    }

    \plush{%
        \innoBanner{Event Store vs. Event Bus}
        \small
        ``Event Store can be represented by any storage of events,
        while Event Bus can also spread them out to the related subsribers.''
    }

    \subcrumbection{Replay}
    \plush[5]{%
        \innoBanner{Event Replay}
        \innoPic{0.7}{replay}
    }

    \subcrumbection{Drawbacks}
    \plush{%
        \innoSection{Less Control than Rq-Rs}
        \innoPic{0.7}{flow}
    }

    \plush{%
        \innoSection{Consistency}
        \innoPic{0.7}{eventual-consistency}
    }

    \plush{%
        \innoSection{Complexity}
        ``For some situations, this separation can be valuable,
        but beware that for most systems ES \& CQRS adds risky complexity.''
    }

    \plush{%
        \innoSection{Event Sourcing vs. CRUD}
        \innoPic{0.7}{cost-scope}
    }

    \subcrumbection{Hex}
    \plush{%
        \innoBanner{Hexagonal Architecture}
        \innoPic{0.8}{hex}
    }
\end{document}