% MIT License
%
% Copyright (c) 2023 Aliaksei Bialiauski
%
% Permission is hereby granted, free of charge, to any person obtaining a copy
% of this software and associated documentation files (the "Software"), to deal
% in the Software without restriction, including without limitation the rights
% to use, copy, modify, merge, publish, distribute, sublicense, and/or sell
% copies of the Software, and to permit persons to whom the Software is
% furnished to do so, subject to the following conditions:
%
% The above copyright notice and this permission notice shall be included in all
% copies or substantial portions of the Software.
%
% THE SOFTWARE IS PROVIDED "AS IS", WITHOUT WARRANTY OF ANY KIND, EXPRESS OR
% IMPLIED, INCLUDING BUT NOT LIMITED TO THE WARRANTIES OF MERCHANTABILITY,
% FITNESS FOR A PARTICULAR PURPOSE AND NONINFRINGEMENT. IN NO EVENT SHALL THE
% AUTHORS OR COPYRIGHT HOLDERS BE LIABLE FOR ANY CLAIM, DAMAGES OR OTHER
% LIABILITY, WHETHER IN AN ACTION OF CONTRACT, TORT OR OTHERWISE, ARISING FROM,
% OUT OF OR IN CONNECTION WITH THE SOFTWARE OR THE USE OR OTHER DEALINGS IN THE
% SOFTWARE.

\documentclass{article}
\usepackage{..//cover}
\usepackage{..//slides}
\usepackage{..//inno}
\usepackage[normalem]{ulem}
\newcommand*\thetitle{SOA}
\newcommand*\thesubtitle{and Microservices}
\begin{document}

    \plush{\defaultInnoTitlePage \innoDisclaimer}

    \innoToc

    \plush{\innoChapter[SOA]{Service-Oriented Architecture (SOA)}}

    \subcrumbection{Microservices}
    \plush[5]{%
        \innoSection{Long time ago}
        \innoPic{0.9}{soa}
    }

    \plush[5]{%
        \innoSection{Microservices}
        \innoPic{0.6}{microservices}\par
        \small
        Microservices are a modern interpretation of service-oriented
        architectures used to build distributed software systems. --- Wikipedia
    }

    \subcrumbection{API}
    \plush[]{%
        \innoSection{API size}
        GitHub RESTful API
        \\
        YouTube
    }

    \plush[5]{%
        \innoBanner{Stateless vs. Stateful Architecture}
        ``A stateless process or application can be understood in isolation. There is no stored knowledge of or reference to past transactions. Each transaction is made as if from scratch for the first time.'' ---
        \href{https://www.redhat.com/en/topics/cloud-native-apps/stateful-vs-stateless}{RedHat}
    }

    \plush{\innoChapter[CAP]{CAP theorem}}

    \subcrumbection{Consistency}
    \plush[4]{%
        \innoSection{Consistency}
        \textbf{ACID vs BASE}
        \\
        \small
        Consistency model - rules that define the order of updates in the system and when these updates become visible to users. --- Wikipedia
    }

    \subcrumbection{Availability}
    \plush[4]{%
        \innoSection{Availability}
        \innolick{$A = \dfrac{E_\text{up}}{E_\text{down} + E_\text{up}}$}
        \\
        \\
        Goal: minimize downtime
        \\
        98\% 99\% 100\%
        \\
        it is not about a number
        \\
        it is about architecture(static) and process(dynamic)
    }

    \subcrumbection{Network Partition}
    \plush[4]{%
        \innoSection{Partition tolerance}
        \small
        Partition tolerance, in the CAP context, means the ability of a data processing system to continue processing data even if a network partition causes communication errors between subsystems.
    }

    \plush{\innoChapter[Containers]{Containers}}

    \subcrumbection{Docker}
    \plush[4]{%
        \innoSection{Docker}
        \innoPic{0.7}{dockerfile}
    }

    \subcrumbection{K8s}
    \plush[4]{%
        \innoSection{Kubernetes}
        \innoPic{0.7}{kubernetes}
    }

    \subcrumbection{SVC Mesh}
    \plush[4]{%
        \innoSection{Service Mesh}
        \innoPic{0.7}{mesh}
    }
\end{document}