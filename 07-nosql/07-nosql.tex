% MIT License
%
% Copyright (c) 2023 Aliaksei Bialiauski
%
% Permission is hereby granted, free of charge, to any person obtaining a copy
% of this software and associated documentation files (the "Software"), to deal
% in the Software without restriction, including without limitation the rights
% to use, copy, modify, merge, publish, distribute, sublicense, and/or sell
% copies of the Software, and to permit persons to whom the Software is
% furnished to do so, subject to the following conditions:
%
% The above copyright notice and this permission notice shall be included in all
% copies or substantial portions of the Software.
%
% THE SOFTWARE IS PROVIDED "AS IS", WITHOUT WARRANTY OF ANY KIND, EXPRESS OR
% IMPLIED, INCLUDING BUT NOT LIMITED TO THE WARRANTIES OF MERCHANTABILITY,
% FITNESS FOR A PARTICULAR PURPOSE AND NONINFRINGEMENT. IN NO EVENT SHALL THE
% AUTHORS OR COPYRIGHT HOLDERS BE LIABLE FOR ANY CLAIM, DAMAGES OR OTHER
% LIABILITY, WHETHER IN AN ACTION OF CONTRACT, TORT OR OTHERWISE, ARISING FROM,
% OUT OF OR IN CONNECTION WITH THE SOFTWARE OR THE USE OR OTHER DEALINGS IN THE
% SOFTWARE.

\documentclass{article}
\usepackage{..//cover}
\usepackage{..//slides}
\usepackage{..//inno}
\usepackage[normalem]{ulem}
\newcommand*\thetitle{NoSQL}
\newcommand*\thesubtitle{vs. SQL}
\newcommand{\lat}{\selectlanguage{english}}
\newcommand{\gre}{\selectlanguage{greek}}
\begin{document}

    \plush{\defaultInnoTitlePage \innoDisclaimer}

    \innoToc

    \plush[5]{%
        \innoBanner{BASE}
        \innoPic{0.8}{base}
    }

    \plush{\innoChapter[Types]{NoSQL Types}}

    \subcrumbection{K-V}
    \plush[5]{%
        \innoSection{Key-Value}
        \innoPic{0.4}{key-value}\par
        \innoPic{0.5}{dynamo}\par
        E.g. Redis, AWS DynamoDB
    }

    \plush{%
        \innoSection{Single Table Design}
        \innoPic{0.7}{single}
    }

    \subcrumbection{Document}
    \plush[5]{%
        \innoSection{Document databases}
        \innoPic{0.5}{document}\par
        E.g. MongoDB
    }

    \subcrumbection{Graph}
    \plush[5]{%
        \innoSection{Graph}
        \innoPic{0.5}{graph}\par
        E.g. Neo4j
    }

    \subcrumbection{Wide}
    \plush[5]{%
        \innoSection{Wide columnar}
        \innoPic{0.8}{columnar}\par
        E.g. Apache Cassandra\par
    }

    \plush{%
        \innoSection{For time-series data and \gre \textit{OLTP}}
        \innoPic{0.7}{timeseries}
    }

    \plush{\innoChapter[Indexes]{Indexes}}
    \plush[5]{%
        \innoBanner{B-tree index}
        Self-balancing tree data structure\par
        insert, delete, search - \gre \textit{O(log(n))}\par
        \innoPic{0.8}{btree}
    }

    \plush[5]{%
        \innoBanner{Log Structured Merge-tree}
        \innoPic{1}{lsm}
    }

    \plush[5]{%
        \innoBanner{LSM vs. B-tree}
        \begin{innoWide}{2}
            \small
            \innoBanner{LSM}
            \begin{itemize}
                \item Google Bigtable
                \item Apache HBase
                \item Apache Cassandra
                \item Influx DB
            \end{itemize}
            \par\columnbreak
            \innoBanner{B-tree}
            \begin{itemize}
                \item MySQL
                \item PostgreSQL
                \item Apache CouchDB
                \item DynamoDB
            \end{itemize}
        \end{innoWide}\par
    }

    \plush{%
        \innoPic{1}{lsm-vs-btree1}
    }

    \plush{%
        \innoPic{1}{lsm-vs-btree2}
    }

    \plush{\innoChapter[Networking]{Database networking}}

    \subcrumbection{Router}
    \plush[5]{%
        \innoBanner{Request Routing}
        \innoPic{0.7}{routing}\par
        E.g. MySQL, PostgreSQL, MongoDB
    }

    \subcrumbection{Gossiping}
    \plush[5]{%
        \innoBanner{Gossip protocol}
        \innoPic{0.7}{gossiping}\par
        E.g. Riak, Cassandra, Dynamo
    }

    \plush{\innoChapter[Properties]{How to Choose the Right Database?}}

    \plush[3]{%
        \innoSection{Durability: Can We Loose Data?}
        \href{https://aws.amazon.com/s3/faqs}{FAQ}: Amazon S3 is designed to provide 99.999999999\% (11 9's) of data durability of objects over a given year. This durability level corresponds to an average annual expected loss of 0.000000001\% of objects. For example, if you store 10,000,000 objects with Amazon S3, you can on average expect to incur a loss of a single object once every 10,000 years.''
    }

    \plush{%
        \innoSection{ACID}
        \nospell{\gre \textbf{\large A}tomicity}: everything or nothing\par
        \nospell{\gre \textbf{\large C}onsistency}: invariants are in place\par
        \nospell{\gre \textbf{\large I}solation}: concurrent or sequential\par
        \nospell{\gre \textbf{\large D}urability}: completed transactions $\rightarrow$ non-volatile memory\par
    }

    \plush{%
        \innoSection{Performance}
        Queries Profiling \& Optimization\par
        Denormalization\par
        Caching\par
    }

    \plush{%
        \innoSection{Scalability}
        Vertical vs. Horizontal Scalability\par
        Sharding vs. Master-Slave Replication\par
        \innoPic{0.7}{sharding}
    }

    \plush{%
        \innoSection{Application Layer Support}
        Is it open source?\par
        How mature is the library?\par
        Is it a thin driver or ORM-ish framework?\par
        Is the API open?\par
    }

    \plush{%
        \innoSection{CAP}
        \innoPic{0.7}{cap}
    }
\end{document}