% MIT License
%
% Copyright (c) 2023 Aliaksei Bialiauski
%
% Permission is hereby granted, free of charge, to any person obtaining a copy
% of this software and associated documentation files (the "Software"), to deal
% in the Software without restriction, including without limitation the rights
% to use, copy, modify, merge, publish, distribute, sublicense, and/or sell
% copies of the Software, and to permit persons to whom the Software is
% furnished to do so, subject to the following conditions:
%
% The above copyright notice and this permission notice shall be included in all
% copies or substantial portions of the Software.
%
% THE SOFTWARE IS PROVIDED "AS IS", WITHOUT WARRANTY OF ANY KIND, EXPRESS OR
% IMPLIED, INCLUDING BUT NOT LIMITED TO THE WARRANTIES OF MERCHANTABILITY,
% FITNESS FOR A PARTICULAR PURPOSE AND NONINFRINGEMENT. IN NO EVENT SHALL THE
% AUTHORS OR COPYRIGHT HOLDERS BE LIABLE FOR ANY CLAIM, DAMAGES OR OTHER
% LIABILITY, WHETHER IN AN ACTION OF CONTRACT, TORT OR OTHERWISE, ARISING FROM,
% OUT OF OR IN CONNECTION WITH THE SOFTWARE OR THE USE OR OTHER DEALINGS IN THE
% SOFTWARE.

\documentclass{article}
\usepackage{..//cover}
\usepackage{..//slides}
\usepackage{..//inno}
\usepackage[normalem]{ulem}
\newcommand*\thetitle{Messaging}
\newcommand*\thesubtitle{in distributed systems}
\begin{document}

    \plush{\defaultInnoTitlePage \innoDisclaimer}

    \innoToc

    \plush{\innoChapter[RRvs.PC]{Request-Response vs Producer-Consumer}}

    \plush{%
        \innoBanner{Request-Response}
        \innoPic{0.5}{rqrs}
    }

    \subcrumbection{Availability}
    \plush[5]{%
        \innoSection{Availability}
        What should the client do when the server is not available?
    }

    \subcrumbection{Performance}
    \plush[5]{%
        \innoSection{Performance}
        What should the client do when the server gets slower?
    }

    \subcrumbection{Backpressure}
    \plush[5]{%
        \innoSection{Backpressure}
        What should the server do when there is a sudden traffic spike?
    }

    \subcrumbection{Retry}
    \plush[5]{%
        \innoSection{Retry}
        What should the client do with failed requests?
    }

    \subcrumbection{Async}
    \plush{%
        \innoSection{Asynchronous messaging}
        \innoPic{0.8}{async}
    }

    \plush{%
        \ff{Availability} \ff{->} Keep messages and send them later\par
        \ff{Performance} \ff{->} Add more servers to parallelize processing\par
        \ff{Backpressure} \ff{->} Keep draining messages at its own pace\par
        \ff{Retry} \ff{->} Re-send failed messages\par
        Also, asynchronous messaging helps to \ff{decouple} client and server.
    }

    \plush{\innoChapter[Patterns]{Messaging Patterns}}

    \subcrumbection{MQ}
    \plush{%
        \innoBanner{Message Queuing}
        \innoPic{0.9}{mq}
        \url{https://aws.amazon.com/sqs}\par
        \url{https://www.rabbitmq.com}
    }

    \subcrumbection{Pub-Sub}
    \plush{%
        \innoBanner{Publish-Subscribe}
        \innoPic{0.8}{pub-sub}
        \url{https://kafka.apache.org}\par
        \url{https://aws.amazon.com/kinesis}
    }

    \plush{%
        \innoPic{1}{arc}
    }

    \subcrumbection{P\&P}
    \plush{%
        \innoBanner{Push model}
        \innoPic{0.7}{push}
    }

    \plush{%
        \innoBanner{Pull model}
        \innoPic{0.6}{pull}
    }

    \plush{%
        \innoBanner{Push vs Pull}
        \innoPic{0.8}{push-vs-pull}
    }

    \subcrumbection{Competing}
    \plush{%
        \innoBanner{Competing Consumers}
        \innoPic{0.8}{competing}
    }

    \subcrumbection{Claim}
    \plush{%
        \innoBanner{Claim Check}
        \innoPic{0.7}{claim}
    }

    \subcrumbection{Retry}
    \plush{%
        \innoBanner{When to retry?}
        \innoPic{0.8}{when1}
    }

    \plush{%
        \innoPic{0.8}{when2}
    }

    \plush{%
        \innoBanner{Exponential backoff}
        \innoPic{0.7}{exp-backoff}
    }

    \plush{%
        \innoBanner{Retry at different times}
        \innoPic{0.7}{jitter}
    }

    \subcrumbection{Partitioning}
    \plush{%
        \innoBanner{Sigle consumer vs Multiple consumers}
        \innoPic{0.8}{single-vs-mul}\par
        Multiple consumers break the order of messages
    }

    \plush{%
        \innoBanner{Data partitioning}
        \innoPic{0.8}{partitioning}\par
        Types: Lookup, Range, Hash
    }

    \plush{%
        \innoSection{Netflix architecture}
        \innoPic{0.8}{netflix}
    }

    \plush{\innoChapter[Internals]{Internals of messaging}}

    \subcrumbection{Ack}
    \plush{%
        \innoBanner{Message Acknowledgement}
        \innoPic{0.8}{ack}
    }

    \plush{%
        \innoBanner{Unsafe ack aka ``fire and forget''}
        \innoPic{0.8}{unsafe}
    }

    \plush{%
        \innoBanner{Safe ack}
        \innoPic{0.7}{safe}\par
        It is up to the broker what safe really means
    }

    \plush{%
        \innoBanner{Failed ack}
        \innoPic{0.7}{failed-ack}
    }

    \subcrumbection{Offset}
    \plush{%
        \innoBanner{Consumer Offsets - Head delete}
        \innoPic{0.6}{head}\par
        RabbitMQ, SQS
    }

    \plush{%
        \innoBanner{Incrementing the pointer}
        \innoPic{0.5}{pointer}\par
        Kafka, Kinesis
    }

    \subcrumbection{Polling}
    \plush{%
        \innoBanner{Short Polling}
        \innoPic{0.8}{short}
    }

    \plush{%
        \innoBanner{Long Polling}
        \innoPic{0.8}{long}
    }

    \subcrumbection{Guarantees}
    \plush{%
        \innoBanner{Delivery Guarantees}
        \ff{at-most-once} -> Messages might be lost\par
        \ff{exactly-once} -> Messages processed only one time\par
        \ff{at-least-once} -> Messages migth be processed multiple times\par
        But failures will happen!
    }

    \plush{%
        \innoQuote{vogels}{
            Failures are a given and everything will eventually fail over time: from routers to hard disks, from operating systems to memory units corrupting TCP packets, from transient errors to permanent failures.
            This is a given, whether you are using the highest quality hardware or lower cost components.
        }{ Dr. Werner Vogels}
    }

    \plush{%
        Most of the messaging systems you will work with will be configured to provide \ff{at-least-once} delivery guarantee
    }
\end{document}