% MIT License
%
% Copyright (c) 2023 Aliaksei Bialiauski
%
% Permission is hereby granted, free of charge, to any person obtaining a copy
% of this software and associated documentation files (the "Software"), to deal
% in the Software without restriction, including without limitation the rights
% to use, copy, modify, merge, publish, distribute, sublicense, and/or sell
% copies of the Software, and to permit persons to whom the Software is
% furnished to do so, subject to the following conditions:
%
% The above copyright notice and this permission notice shall be included in all
% copies or substantial portions of the Software.
%
% THE SOFTWARE IS PROVIDED "AS IS", WITHOUT WARRANTY OF ANY KIND, EXPRESS OR
% IMPLIED, INCLUDING BUT NOT LIMITED TO THE WARRANTIES OF MERCHANTABILITY,
% FITNESS FOR A PARTICULAR PURPOSE AND NONINFRINGEMENT. IN NO EVENT SHALL THE
% AUTHORS OR COPYRIGHT HOLDERS BE LIABLE FOR ANY CLAIM, DAMAGES OR OTHER
% LIABILITY, WHETHER IN AN ACTION OF CONTRACT, TORT OR OTHERWISE, ARISING FROM,
% OUT OF OR IN CONNECTION WITH THE SOFTWARE OR THE USE OR OTHER DEALINGS IN THE
% SOFTWARE.

\documentclass{article}
\usepackage{..//cover}
\usepackage{..//slides}
\usepackage{..//inno}
\usepackage[normalem]{ulem}
\newcommand*\thetitle{IO}
\newcommand*\thesubtitle{vs. NIO}
\begin{document}

    \plush{\defaultInnoTitlePage \innoDisclaimer}

    \innoToc

    \plush{\innoChapter[Threads]{Thread Management}}

    \subcrumbection{Socket}
    \plush[5]{%
        \innoSection{Socket}
        \innoPic{0.8}{socket}
    }

    \plush[5]{%
        \innoSection{blocking socket vs. non-blocking socket}
        \begin{innoWide}{2}
            \small
            \innoBanner{blocking}
            thread is suspended until read/write from/to socket completes
            \par\columnbreak
            \innoBanner{non-blocking}
            thread reads data available in the socket buffer and does not wait for the remaining data to arrive
        \end{innoWide}
    }

    \subcrumbection{Connection}
    \plush[5]{%
        \innoSection{Thread per connection}
        \innoPic{0.8}{connection}
    }

    \subcrumbection{Rq}
    \plush[5]{%
        \innoSection{Thread per request}
        \innoPic{0.8}{rq}\par
        \url{https://tomcat.apache.org}
    }

    \subcrumbection{Loop}
    \plush[5]{%
        \innoSection{Event Loop}
        \innoPic{0.8}{loop}\par
        \url{https://netty.io}
    }

    \subcrumbection{CPU}
    \plush[5]{%
        \innoSection{Concurrency vs. Parallelism}
        \small
        At any given time 1 CPU can run only thread. Parallel execution happens when we have multiple CPUs.
    }

    \plush[5]{%
        \innoBanner{IO vs. CPU bounds}
        Event loop is good for IO-bound workload (e.g. large files).\par
        Thread per request is good for CPU-bound workload.
    }

    \plush{\innoChapter[Sync]{Synchronous communication}}
    \subcrumbection{REST}
    \plush[5]{%
        \innoSection{REST communication}
        \innoSnippet[\scriptsize]{RestExample.java}
    }

    \subcrumbection{Feign}
    \plush[5]{%
        \innoSection{Open Feign - Declarative REST Client}
        \innoSnippet[\scriptsize]{StoreClient.java}
        \url{https://docs.spring.io/spring-cloud-openfeign/docs/current/reference/html/}
    }

    \subcrumbection{GraphQL}
    \plush[5]{%
        \innoSection{GraphQL}
        \innoPic{0.8}{graphql-vs-rest}
    }

    \plush[5]{%
        \innoBanner{GraphQL Schema}
        \innoSnippet[\scriptsize]{schema.graphqls}
    }

    \plush[5]{%
        \innoBanner{GraphQL Query}
        \innoSnippet[\scriptsize]{query.graphql}
    }

    \plush[5]{%
        \innoBanner{GraphQL Mutation}
        \innoSnippet[\scriptsize]{mutation.graphql}
    }

    \plush[5]{%
        \innoBanner{Spring Controllers}
        \innoSnippet[\scriptsize]{PostController.java}
    }

    \plush{%
        \url{https://graphql.org/learn/}
    }

    \plush{\innoChapter[Streams]{Reactive Streams}}

    \subcrumbection{Manifesto}
    \plush[5]{%
        \innoSection{Reactive Manifesto}
        \innoPic{0.7}{manifesto}\par
        \url{https://www.reactivemanifesto.org}
    }

    \subcrumbection{Back}
    \plush{%
        \innoSection{Backpressure}
        \small
        In Reactive Streams, backpressure also defines how to regulate the transmission of stream elements.\par
        \innoPic{0.7}{backpressure}
    }

    \subcrumbection{Spec}
    \plush{%
        \innoSection{Reactive Streams}
        \small
        ``Reactive Streams is an initiative to provide a standard for asynchronous stream processing with non-blocking back pressure. This encompasses efforts aimed at runtime environments (JVM and JavaScript) as well as network protocols.''\par
        --- \href{https://www.reactive-streams.org}{Reactive Streams website}
    }

    \plush{%
        \innoBanner{Publisher}
        \innoSnippet[\scriptsize]{Publisher.java}
    }

    \plush{%
        \innoBanner{Subscriber}
        \innoSnippet[\scriptsize]{Subscriber.java}
    }

    \plush{%
        \innoBanner{Subscription}
        \innoSnippet[\scriptsize]{Subscription.java}
    }

    \plush{%
        \innoBanner{Processor}
        \innoSnippet[\scriptsize]{Processor.java}
    }

    \plush{%
        \innoBanner{Pub-Sub Flow}
        \innoPic{0.8}{reactive-streams-flow}
    }

    \plush{%
        \innoBanner{Implementations}
        \url{https://github.com/ReactiveX/RxJava}\par
        \url{https://github.com/eclipse-vertx/vert.x}\par
        \url{https://github.com/smallrye/smallrye-mutiny}\par
        \url{https://github.com/reactor/reactor-core}
    }

    \plush{\innoChapter[Reactor]{Project Reactor}}
    \subcrumbection{Reactor}
    \plush[5]{%
        \innoSection{Project Reactor}
        \small
        ``Well-suited for a microservices architecture, Reactor offers backpressure-ready network engines for HTTP (including Websockets), TCP, and UDP.''\par
        ``Reactor offers two reactive and composable APIs, contains 2 publishers: Flux [N] and Mono [0|1], which extensively implement Reactive Extensions.''\par
        --- \href{https://projectreactor.io}{Reactor website}
    }

    \subcrumbection{Mono}
    \plush[5]{%
        \innoBanner{Mono, publisher with asynchronous 0--1 result.}
        \innoPic{0.8}{mono}
    }

    \plush{%
        \innoSnippet[\scriptsize]{MonoExample.java}
    }

    \subcrumbection{Flux}
    \plush[5]{%
        \innoBanner{Flux, publisher with asynchronous sequence of 0--N items.}
        \innoPic{0.8}{flux}
    }

    \plush{%
        \innoSnippet[\scriptsize]{FluxExample.java}
    }

    \subcrumbection{Virtues}
    \plush{%
        \innoSection{Immutability}
        \innoSnippet[\scriptsize]{ImmutabilityExample.java}
    }

    \plush{%
        \innoSection{Declarativity}
    }

    \subcrumbection{Other}
    \plush{%
        \innoBanner{Other reactor projects}
        \url{https://github.com/reactor/reactor-kafka}\par
        \url{https://github.com/reactor/reactor-netty}\par
    }

    \subcrumbection{Spring}
    \plush[5]{%
        \innoSection{Spring Reactive}
        Spring supports reactive programming using Project Reactor.\par
        Servlet Web -> Reactive Web\par
        JDBC -> R2DBC\par
        Imperative -> Declarative
    }

    \plush[5]{%
        \innoSnippet[\scriptsize]{UserRepository.java}
        \innoSnippet[\scriptsize]{Users.java}
    }

    \plush{%
        \innoSnippet[\scriptsize]{UserRoutes.java}
    }
\end{document}